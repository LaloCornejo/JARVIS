\documentclass[12pt,a4paper]{article}
\usepackage[utf8]{inputenc}
\usepackage[spanish]{babel}
\usepackage{geometry}
\usepackage{graphicx}
\usepackage{hyperref}
\usepackage{booktabs}
\usepackage{enumitem}
\usepackage{xcolor}
\usepackage{listings}
\usepackage{fancyhdr}
\usepackage{titlesec}
\usepackage{tocloft}

% Configuración de márgenes
\geometry{margin=2.5cm}

% Configuración de colores
\definecolor{primary}{RGB}{0, 102, 204}
\definecolor{secondary}{RGB}{51, 51, 51}
\definecolor{accent}{RGB}{255, 153, 0}

% Configuración de hyperlinks
\hypersetup{
    colorlinks=true,
    linkcolor=primary,
    filecolor=magenta,
    urlcolor=blue,
    citecolor=primary
}

% Configuración de headers
\pagestyle{fancy}
\fancyhf{}
\rhead{JARVIS - Asistente Personal con IA}
\lhead{Documento de Investigación}
\cfoot{\thepage}

% Configuración de listings para código
\lstset{
    basicstyle=\ttfamily\small,
    breaklines=true,
    frame=single,
    backgroundcolor=\color{gray!10}
}

\begin{document}

% Portada
\begin{titlepage}
    \centering
    \vspace*{2cm}
    
    {\Huge\bfseries DOCUMENTO DE DEFINICIÓN\\DEL TEMA DE INVESTIGACIÓN\par}
    \vspace{1.5cm}
    
    {\LARGE\textcolor{primary}{Proyecto JARVIS}\par}
    \vspace{0.5cm}
    {\Large Asistente Personal con Inteligencia Artificial\par}
    
    \vspace{2cm}
    
    \includegraphics[width=0.3\textwidth]{example-image-a}
    
    \vspace{2cm}
    
    {\large\textbf{Asignatura:} Metodologia de la Investigacion\par}
    \vspace{0.5cm}    
    \vfill
    
    {\large Febrero 2026\par}
\end{titlepage}

% Índice
\tableofcontents
\newpage

% ==========================================
\section{Lista de Temas de Interés (Relacionados con IA y Asistentes)}
% ==========================================

Los siguientes temas representan áreas de interés relacionadas con el desarrollo de asistentes personales basados en Inteligencia Artificial, cada uno abordando diferentes aspectos tecnológicos y de aplicación:

\begin{enumerate}[label=\arabic*.]
    \item \textbf{Implementación de Procesamiento de Lenguaje Natural (NLP) en entornos locales}
    \begin{itemize}
        \item Uso de modelos de lenguaje locales (Ollama, Whisper) para procesamiento de voz y texto
        \item Transcripción de voz a texto mediante Faster-Whisper con soporte CUDA
        \item Generación de respuestas naturales sin dependencia de servicios en la nube
    \end{itemize}
    
    \item \textbf{Automatización de tareas rutinarias mediante agentes inteligentes}
    \begin{itemize}
        \item Sistema multi-agente con roles especializados (código, investigación, creatividad, planificación)
        \item Orquestación de agentes para tareas complejas que requieren múltiples competencias
        \item Ejecución autónoma de flujos de trabajo predefinidos
    \end{itemize}
    
    \item \textbf{Integración de modelos de lenguaje (LLM) en la organización de agendas personales}
    \begin{itemize}
        \item Conexión con calendarios (Google Calendar) para gestión de eventos
        \item Integración con servicios de productividad (Obsidian, Gmail, Discord, Telegram, WhatsApp)
        \item Monitoreo de tiempo en pantalla y análisis de productividad
    \end{itemize}
    
    \item \textbf{Seguridad y privacidad de datos en asistentes virtuales domésticos}
    \begin{itemize}
        \item Enfoque ``local-first'' para procesamiento de datos sensibles
        \item Sistema de permisos y aprobación para herramientas críticas
        \item Memoria encriptada para almacenamiento seguro de información personal
    \end{itemize}
    
    \item \textbf{Sistemas de memoria semántica y episódica para asistentes de IA}
    \begin{itemize}
        \item Memoria vectorial con LanceDB para búsqueda semántica
        \item Memoria episódica para recuperación de conversaciones y experiencias pasadas
        \item Grafos de conocimiento para relaciones entre entidades
    \end{itemize}
    
    \item \textbf{Interfaces multimodales de interacción humano-computadora}
    \begin{itemize}
        \item Detección de palabra de activación (``Hey JARVIS'')
        \item Síntesis de voz con XTTS para respuestas naturales
        \item Integración con navegadores web mediante Playwright MCP
    \end{itemize}
\end{enumerate}

% ==========================================
\section{Selección y Jerarquización}
% ==========================================

\subsection{1. Asistente IA para la gestión autónoma de productividad (Alta prioridad)}

\textbf{Motivación:} El proyecto JARVIS implementa un sistema completo de gestión de productividad que incluye:

\begin{itemize}
    \item \textbf{Más de 50 herramientas integradas} para control del sistema, archivos, web, música, calendario, notas, email, y más
    \item \textbf{Sistema de memoria avanzado} con memoria semántica, episódica y grafos de conocimiento
    \item \textbf{Monitoreo de tiempo en pantalla} con análisis de productividad por aplicación y categoría
    \item \textbf{Integración con servicios de comunicación} (Discord, Telegram, WhatsApp, Gmail)
    \item \textbf{Gestión de calendarios y notas} (Google Calendar, Obsidian)
\end{itemize}

\textbf{Viabilidad técnica:} El sistema ya está implementado con una arquitectura modular basada en Python 3.11+, con soporte para múltiples backends de LLM (Ollama, Gemini, NVIDIA API).

\subsection{2. Interfaz de control de dispositivos mediante comandos de voz inteligentes (Alta prioridad)}

\textbf{Motivación:} JARVIS implementa un sistema completo de interacción por voz que incluye:

\begin{itemize}
    \item \textbf{Detección de palabra de activación} mediante OpenWakeWord con modelo ``hey\_jarvis''
    \item \textbf{Transcripción de voz} con Faster-Whisper optimizado para CUDA
    \item \textbf{Síntesis de voz natural} mediante servidor XTTS con múltiples voces
    \item \textbf{Detección de actividad de voz (VAD)} con Silero para segmentación precisa
    \item \textbf{Control del sistema} (volumen, aplicaciones, comandos de terminal)
\end{itemize}

\textbf{Afinidad con el hardware:} El sistema soporta aceleración GPU mediante CUDA para procesamiento de audio en tiempo real.

\subsection{3. Sistema multi-agente para tareas complejas (Media prioridad)}

\textbf{Motivación:} JARVIS implementa una arquitectura de agentes especializados:

\begin{itemize}
    \item \textbf{CodeReviewAgent:} Análisis y revisión de código
    \item \textbf{ResearchAgent:} Investigación y búsqueda de información
    \item \textbf{CreativeAgent:} Generación de contenido creativo
    \item \textbf{PlanningAgent:} Planificación y gestión de proyectos
    \item \textbf{AgentOrchestrator:} Coordinación y enrutamiento inteligente de solicitudes
\end{itemize}

\textbf{Formación relacionada:} Este enfoque se relaciona con el análisis de datos y la inteligencia artificial aplicada a la automatización de procesos.

% ==========================================
\section{Revisión de Información y Antecedentes}
% ==========================================

\subsection{Autores y Desarrolladores de Referencia}

Se identifica que existe una amplia base de autores y desarrolladores en el campo de la Inteligencia Artificial:

\begin{table}[h]
\centering
\begin{tabular}{lll}
\toprule
\textbf{Organización} & \textbf{Tecnología} & \textbf{Aplicación en JARVIS} \\
\midrule
OpenAI & GPT, Whisper & Transcripción de voz \\
Google & Gemini, Transformers & Backend LLM alternativo \\
Meta & LLaMA, Sentence-BERT & Embeddings semánticos \\
NVIDIA & CUDA, APIs de inferencia & Aceleración de modelos \\
Anthropic & Claude & Modelos de razonamiento \\
Hugging Face & Transformers, Datasets & Modelos preentrenados \\
\bottomrule
\end{tabular}
\caption{Tecnologías y su aplicación en JARVIS}
\end{table}

\subsection{Documentación Técnica Consultada}

La investigación se apoya en:

\begin{enumerate}
    \item \textbf{Arquitecturas de Transformadores:}
    \begin{itemize}
        \item Faster-Whisper para transcripción de audio
        \item Sentence-Transformers (all-MiniLM-L6-v2) para embeddings semánticos
        \item Modelos de lenguaje local via Ollama (qwen3, llama3)
    \end{itemize}
    
    \item \textbf{Interacción Humano-Computadora (HCI) mediante voz:}
    \begin{itemize}
        \item OpenWakeWord para detección de palabra de activación
        \item Silero-VAD para detección de actividad de voz
        \item XTTS para síntesis de voz natural
    \end{itemize}
    
    \item \textbf{Frameworks de código abierto:}
    \begin{itemize}
        \item Model Context Protocol (MCP) para extensibilidad
        \item LanceDB para almacenamiento de vectores
        \item FastAPI/WebSockets para comunicación en tiempo real
        \item Textual para interfaces de terminal enriquecidas
    \end{itemize}
\end{enumerate}

\subsection{Protocolos y Estándares}

\begin{itemize}
    \item \textbf{MCP (Model Context Protocol):} Protocolo estándar para conexión con servidores externos
    \begin{itemize}
        \item Servidor de sistema de archivos
        \item Servidor de GitHub
        \item Servidor de memoria
        \item Servidor SQLite
        \item Servidor Playwright (automatización web)
        \item Servidor Context7 y Exa (búsqueda semántica)
    \end{itemize}
    
    \item \textbf{APIs de comunicación:}
    \begin{itemize}
        \item Telegram Bot API
        \item Discord API
        \item WhatsApp Bailey (no oficial)
        \item Gmail API
    \end{itemize}
\end{itemize}

% ==========================================
\section{Propósito de la Investigación}
% ==========================================

\subsection{Respuesta y Justificación}

El desarrollo del asistente JARVIS tiene como propósito principal \textbf{optimizar la gestión del tiempo y la productividad del usuario} mediante un sistema que:

\begin{enumerate}
    \item \textbf{No solo recibe órdenes, sino que aprende de los hábitos:}
    \begin{itemize}
        \item Sistema de aprendizaje automático (SelfImprovement) que registra comandos, respuestas y retroalimentación
        \item Análisis de patrones de uso mediante PatternAnalyzer
        \item Detección de anomalías con AnomalyDetector
        \item Motor de sugerencias inteligentes (SmartSuggestionEngine)
    \end{itemize}
    
    \item \textbf{Reduce la carga cognitiva:}
    \begin{itemize}
        \item Procesamiento de lenguaje natural para comandos intuitivos
        \item Automatización de tareas repetitivas mediante workflows
        \item Sistema de triggers temporales para recordatorios y acciones programadas
        \item Respuestas proactivas basadas en contexto
    \end{itemize}
    
    \item \textbf{Centraliza la información en una sola interfaz inteligente:}
    \begin{itemize}
        \item Más de 50 herramientas integradas en un único punto de acceso
        \item Memoria unificada (semántica, episódica, procedimental)
        \item Interfaces múltiples: voz, TUI, WebSocket, Telegram
        \item Sincronización con servicios externos (calendario, notas, email, repositorios)
    \end{itemize}
\end{enumerate}

\subsection{Objetivos Específicos}

\begin{enumerate}
    \item Proporcionar una interfaz de voz natural y responsiva con latencia mínima
    \item Implementar un sistema de memoria persistente que permita continuidad entre sesiones
    \item Ofrecer integración transparente con servicios de productividad existentes
    \item Garantizar la privacidad mediante procesamiento local de datos sensibles
    \item Permitir extensibilidad mediante el protocolo MCP para nuevas funcionalidades
\end{enumerate}

% ==========================================
\section{Delimitación del Tema}
% ==========================================

\subsection{Alcance del Proyecto}

Para evitar un alcance excesivamente amplio que comprometa los recursos y el tiempo disponible, el tema final se define de la siguiente manera:

\begin{center}
\fbox{
\parbox{0.9\textwidth}{
\centering
\textbf{\large Tema Seleccionado:}\\[0.5cm]
\textit{``Desarrollo de un asistente personal basado en Inteligencia Artificial con capacidades de procesamiento de voz local, memoria semántica persistente y sistema multi-agente para la automatización de tareas de productividad personal''}
}
}
\end{center}

\subsection{Características Delimitadas del Sistema JARVIS}

\subsubsection{Componentes Principales}

\begin{enumerate}
    \item \textbf{Núcleo de Procesamiento (core/)}
    \begin{itemize}
        \item \texttt{assistant.py}: Orquestador principal del asistente
        \item \texttt{voice/}: Módulos de STT, TTS, VAD y detección de wake word
        \item \texttt{memory/}: Memoria vectorial, semántica y episódica
        \item \texttt{llm/}: Clientes para Ollama, Gemini, NVIDIA
        \item \texttt{learning/}: Sistema de auto-mejora y análisis de patrones
    \end{itemize}
    
    \item \textbf{Sistema de Herramientas (tools/)}
    \begin{itemize}
        \item \texttt{system/}: Control del sistema
        \item \texttt{web/}: Búsqueda web, fetch de URLs
        \item \texttt{integrations/}: Spotify, Discord, Telegram, GitHub, Docker
        \item \texttt{memory/}: Almacenamiento y recuperación de memoria
        \item \texttt{code/}: Ejecución de Python y shell
    \end{itemize}
    
    \item \textbf{Sistema Multi-Agente (agents/)}
    \begin{itemize}
        \item \texttt{orchestrator/}: Coordinación y enrutamiento
        \item \texttt{specialized/}: Agentes de código, investigación, creatividad, planificación
    \end{itemize}
    
    \item \textbf{Integraciones Externas}
    \begin{itemize}
        \item MCP Servers: filesystem, GitHub, memory, SQLite, Playwright
        \item APIs: Telegram Bot, Discord, WhatsApp, Gmail, Calendar
    \end{itemize}
\end{enumerate}

\subsubsection{Tecnologías Específicas}

\begin{table}[h]
\centering
\begin{tabular}{lll}
\toprule
\textbf{Categoría} & \textbf{Tecnología} & \textbf{Versión/Modelo} \\
\midrule
Lenguaje & Python & 3.11+ \\
LLM Local & Ollama & qwen3:1.7b, qwen3-vl:8b \\
LLM Cloud & NVIDIA API & moonshotai/kimi-k2.5 \\
STT & Faster-Whisper & base.en (CUDA) \\
TTS & XTTS Server & Puerto 8020 \\
Wake Word & OpenWakeWord & hey\_jarvis \\
VAD & Silero-VAD & 0.0.2 \\
Embeddings & Sentence-Transformers & all-MiniLM-L6-v2 \\
Vector DB & LanceDB & 0.6.0+ \\
Framework Web & FastAPI + WebSockets & 0.104.0+ \\
\bottomrule
\end{tabular}
\caption{Stack tecnológico de JARVIS}
\end{table}

\subsubsection{Limitaciones Definidas}

\begin{itemize}
    \item \textbf{Plataforma objetivo:} Windows (con compatibilidad parcial Linux/macOS)
    \item \textbf{Requisitos de hardware:} GPU NVIDIA con CUDA para rendimiento óptimo
    \item \textbf{Idioma principal:} Inglés (modelo STT base.en)
    \item \textbf{Modo de operación:} Local-first con opciones de backend en la nube
    \item \textbf{Usuario objetivo:} Usuario individual 
\end{itemize}

% ==========================================
\section{Arquitectura del Sistema}
% ==========================================

\subsection{Diagrama de Componentes}

\begin{verbatim}
+----------------------------------------------------------+
|                      JARVIS SYSTEM                        |
+----------------------------------------------------------+
|                                                          |
|  +----------------+    +----------------+                |
|  |   Voice Input  |    |   LLM Backend  |                |
|  +----------------+    +----------------+                |
|  | - Wake Word    |    | - Ollama       |                |
|  | - STT (Whisper)|    | - NVIDIA API   |                |
|  | - VAD (Silero) |    | - Gemini       |                |
|  +-------+--------+    +-------+--------+                |
|          |                     |                         |
|          v                     v                         |
|  +-------------------------------------------+           |
|  |           VoiceAssistant (Core)           |           |
|  +-------------------------------------------+           |
|  | - State Management                        |           |
|  | - Tool Orchestration                      |           |
|  | - Memory Integration                      |           |
|  | - Response Generation                     |           |
|  +--------------------+----------------------+           |
|                       |                                  |
|          +------------+------------+                     |
|          |            |            |                     |
|          v            v            v                     |
|  +----------+  +------------+  +------------+            |
|  |  Tools   |  |   Memory   |  |   Agents   |            |
|  +----------+  +------------+  +------------+            |
|  | 50+ tools|  | - Semantic |  | - Code     |            |
|  | - System |  | - Episodic |  | - Research |            |
|  | - Web    |  | - Vector   |  | - Creative |            |
|  | - Integr.|  | - Knowledge|  | - Planning |            |
|  +----------+  +------------+  +------------+            |
|                                                          |
+----------------------------------------------------------+
\end{verbatim}

\subsection{Flujo de Procesamiento de Voz}

\begin{enumerate}
    \item \textbf{Detección de Wake Word:} OpenWakeWord detecta ``Hey JARVIS''
    \item \textbf{Captura de Audio:} Stream de audio a 16kHz, chunks de 512 samples
    \item \textbf{Detección de Silencio:} VAD determina fin de utterance
    \item \textbf{Transcripción:} Faster-Whisper convierte audio a texto
    \item \textbf{Procesamiento:} LLM genera respuesta con acceso a herramientas
    \item \textbf{Ejecución:} Herramientas se ejecutan según sea necesario
    \item \textbf{Síntesis:} XTTS genera audio de respuesta
    \item \textbf{Reproducción:} Audio se reproduce con soporte de interrupción
\end{enumerate}

% ==========================================
\section{Conclusiones Preliminares}
% ==========================================

El proyecto JARVIS representa una implementación comprehensiva de un asistente personal con IA que:

\begin{enumerate}
    \item \textbf{Aborda múltiples áreas de investigación} en IA aplicada:
    \begin{itemize}
        \item Procesamiento de lenguaje natural (NLP/NLU)
        \item Reconocimiento y síntesis de voz
        \item Sistemas multi-agente
        \item Memoria y razonamiento
    \end{itemize}
    
    \item \textbf{Implementa soluciones prácticas} para problemas reales:
    \begin{itemize}
        \item Gestión de productividad personal
        \item Automatización de tareas rutinarias
        \item Centralización de servicios digitales
    \end{itemize}
    
    \item \textbf{Utiliza tecnologías de vanguardia} manteniendo:
    \begin{itemize}
        \item Privacidad mediante procesamiento local
        \item Extensibilidad mediante protocolos abiertos (MCP)
        \item Rendimiento mediante aceleración GPU
    \end{itemize}
\end{enumerate}

% ==========================================
\section{Referencias y Recursos}
% ==========================================

\subsection{Repositorios y Documentación}

\begin{itemize}
    \item OpenWakeWord: \url{https://github.com/dscripka/openWakeWord}
    \item Faster-Whisper: \url{https://github.com/SYSTRAN/faster-whisper}
    \item Ollama: \url{https://ollama.ai/}
    \item LanceDB: \url{https://lancedb.github.io/lancedb/}
    \item Model Context Protocol: \url{https://modelcontextprotocol.io/}
    \item Sentence-Transformers: \url{https://www.sbert.net/}
\end{itemize}

\end{document}
